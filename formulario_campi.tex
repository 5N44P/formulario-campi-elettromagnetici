\documentclass{article}
\usepackage[utf8]{inputenc}
\usepackage[italian]{babel}
\usepackage{geometry}
\usepackage{amsfonts} 
\usepackage{ccicons}
\usepackage{url}
\usepackage{hyperref}
\usepackage{amsmath}
\usepackage{cellspace}
\usepackage{setspace}
\setlength\cellspacetoplimit{4pt}
\setlength\cellspacebottomlimit{4pt}


\geometry{left = 2 cm, right = 2 cm, bottom = 2 cm, top = 2 cm}
\setlength{\parindent}{0em}
\hypersetup{
	colorlinks=true,
	urlcolor=blue,
    colorlinks,
    linkcolor={black},
    citecolor={black}
}

\pagestyle{plain}
\pagenumbering{gobble}

\title{\Huge Formulario di Campi Elettromagnetici}
\author{\LARGE Lorenzo Rossi}
\date{\LARGE Anno Accademico 2019/2020}


\begin{document}
\maketitle

\vspace{18em}

\large
\begin{doublespacing}
\centerline{Email: \href{mailto://lorenzo14.rossi@mail.polimi.it}{lorenzo14.rossi@mail.polimi.it}}
\centerline{GitHub: \url{https://github.com/lorossi/formulario-campi-elettromagnetici}}

\vspace{18em}
\centerline{Quest'opera è distribuita con Licenza Creative Commons Attribuzione}
\centerline{Non commerciale 4.0 Internazionale \ccbynceu}
\centerline{Versione aggiornata al 15/06/2020}
\end{doublespacing}
\newpage


\pagenumbering{roman}
\tableofcontents
\clearpage
\pagenumbering{arabic}
\newpage

\section{Riguardo al formulario}
Quest'opera è distribuita con Licenza Creative Commons \newline Attribuzione Non commerciale 4.0 Internazionale \ccbynceu \newline
Questo formulario verrà espanso (ed, eventualmente, corretto) periodicamente fino a fine corso. \newline
Link repository di GitHub: \url{https://github.com/lorossi/formulario-campi-elettromagnetici} \newline
L'ultima versione può essere scaricata direttamente \href{https://github.com/lorossi/formulario-campi-elettromagnetici/blob/master/formulario_campi.pdf}{da questo link} premendo poi su "Download".

\section{Richiami di base}
\subsection{Trigonometria}
\begin{itemize}
	\item Teorema di Carnot \( a^2 = b^2 + c^2 - 2bc\cos(\alpha)\) con \(\alpha\) angolo compreso tra \(b\) e \(c\) 
\end{itemize}

\subsection{Numeri complessi}
\begin{itemize}
	\item Radice quadrata di numeri complessi \(z \in \mathbb{C}\), \\ \( z = \alpha + j \beta = r e^{j \theta} \rightarrow \sqrt{z} = \sqrt[n]{z} \left( \cos{\dfrac{\theta + 2k\pi}{n}} + j \sin{\dfrac{\theta + 2 k \pi}{n}} \right) \) \newline con \(k \in \{0, 1, \cdots, n-1 \} \)
	\item Parte reale di un numero complesso \(x \in \mathbb{C}\), \(Re[x] = \dfrac{x + x^{*}}{2} \)
	\item Inverso della parte reale: siano \(z = \alpha + j \beta \), \(y = \dfrac{1}{z} \rightarrow \operatorname{Re}(y) \neq \dfrac{1}{\alpha}\), \(\operatorname{Re}(y)  = \dfrac{\alpha}{\alpha^2 + \beta^2}\)
\end{itemize}

\subsection{Decibel e Neper}
\begin{itemize}
	\item Decibel
	\begin{itemize}
		\item Adimensionali
		\begin{itemize}
			\item Corrente, tensione, campo ... \( 20 \log\left(\dfrac{x}{x_0}\right) \)
			\item Potenza, densità di potenza ... \( 10 \log\left(\dfrac{x}{x_0}\right) \)
			\item \(x_0\) valore di riferimento
		\end{itemize}
		\item Potenza 
		\begin{itemize}
			\item \(1 dBw = 10 \log\left(\dfrac{P}{1w}\right) \)
			\item \(1 dBm = 10 \log\left(\dfrac{P}{1mw}\right) \)
		\end{itemize}
		\item Tensione 
		\begin{itemize}
			\item \(1 dBv = 20 \log\left(\dfrac{P}{1v}\right) \)
			\item \(1 dB \mu v = 20 \log\left(\dfrac{P}{1 \mu v}\right) \)
		\end{itemize}
	\end{itemize}
	\item Neper: si usa il logaritmo naturale (\( \ln \)) al posto del logaritmo in base 10 (\(\log\)).
	\item Conversione
	\begin{itemize}
		\item Neper \(\rightarrow\) Decibel \( \; \alpha_{db} = \alpha_{Np} \cdot 8.686 \)
		\item Decibel\(\rightarrow\) Neper \( \; \alpha_{Np} = \dfrac{\alpha_{Db}}{8.686} \)
	\end{itemize}
\end{itemize}

\subsection{Teoremi fondamentali}
\begin{itemize}
	\item Teorema di Stokes (rotore)
	\begin{itemize}
		\item Si applica a campi vettoriali su una linea chiusa orientabile ed orientata in modo coerente alla normale della superficie tramite regola della mano destra
		\item \( \displaystyle \oint_s \vec{F} \cdot d\vec{l} = \iint_\Omega \nabla \cdot \vec{F} d\Omega \)
	\end{itemize}	 
	\item Teorema di Gauss (divergenza)
	\begin{itemize}
		\item Si applica ad un campo vettoriale su una superficie chiusa semplice ed orientata con bordo regolare \( \Omega \)
		\item \( \displaystyle \iiint_V \nabla \cdot \vec{F} d\Omega = \iint_\Omega \vec{F} \cdot d\vec{S} = \iint_\Omega \left( \vec{F} \cdot \vec{n} \right) dS \)
	\end{itemize}
\end{itemize}

\subsection{Sistemi di coordinate}
\begin{itemize}
	\item Coordinate cartesiane
	\begin{itemize}
		\item Elemento di spostamento infinitesimo \( dl = dx \vec{u_x} + dy \vec{u_y} + dz \vec{u_z} \)
		\item Elemento di volume \( dV = dx dy dz \)
	\end{itemize}
	
	\item Coordinate cilindriche
	\begin{itemize}
		\item Elemento di spostamento infinitesimo \( dl = d\rho \vec{u_\rho} + \rho d\phi \vec{u_\phi} + dz \vec{u_z} \)
		\item Elemento di volume \( dV = \rho d\rho d\phi d_z\)
	\end{itemize}
	
	\item Coordinate polari
	\begin{itemize}
		\item Elemento di spostamento infinitesimo \( dl = dr \vec{u_r} + R d\theta \vec{u_\theta} + R \sin\theta d\phi \vec{u_\phi} \)
		\item Elemento di volume \( dV = R^2 \sin\theta d\theta d\phi dR \)
	\end{itemize}
\end{itemize}

\subsection{Operatori differenziali}
\begin{itemize}
	\item Gradiente
	\begin{itemize}
		\item campo scalare \(\rightarrow\) campo vettoriale
		\item \( \nabla \, F(x, y, z) = \dfrac{\partial F}{\partial x} \vec{u_x} +  \dfrac{\partial F}{\partial y} \vec{u_y} + \dfrac{\partial F}{\partial z} \vec{u_z}\)
	\end{itemize}
	\item Divergenza
	\begin{itemize}
		\item 
		\item \( \nabla \cdot \vec{F}(x, y, z) = \dfrac{\partial F_x}{\partial x} +  \dfrac{\partial F_y}{\partial y} + \dfrac{\partial F_x}{\partial z} \)
	\end{itemize}
	\item Rotore
	\begin{itemize}
		\item campo vettoriale \(\rightarrow\) campo vettoriale
		\item \( \nabla \times \vec{F} = \left( \dfrac{\partial F_z}{\partial y} -  \dfrac{\partial F_y}{\partial z} \right) \vec{u_x} + \left( \dfrac{\partial F_c}{\partial z} -  \dfrac{\partial F_z}{\partial x} \right) \vec{u_y} + \left( \dfrac{\partial F_y}{\partial x} -  \dfrac{\partial F_x}{\partial y} \right) \vec{u_z} \)
	\end{itemize}
\end{itemize}

\subsection{Equazioni di Maxwell nel vuoto}
\begin{enumerate}
	\item Legge di Gauss elettrica \( \nabla \cdot \vec{E} = \dfrac{\rho}{\epsilon_0} \)
	\item Legge di Gauss magnetica \( \nabla \cdot \vec{B} = 0 \)
	\item Legge di Faraday \( \nabla \times \vec{E} = - \dfrac{\partial \vec{B}}{\partial t} \)
	\item Legge di Ampere-Maxwell \( \nabla \times \vec{B} = \mu_0 \vec{J} + \mu_0 \epsilon_0 \dfrac{\partial \vec{E}}{\partial t} \)
\end{enumerate}

\newpage

\section{Elettrostatica}
\begin{itemize}
	\item Legge di Coulomb \( \vec{F} = \dfrac{Qq}{4 \pi \epsilon_0 R^2} \vec{a_r} = \dfrac{Qq}{4 \pi \epsilon_0} \cdot \dfrac{(\vec{R_q} - \vec{R_p})}{| \vec{R_q} - \vec{R_p} | ^3 } \)
	\item Legge di Gauss \( \nabla \cdot \vec{D} = \rho_\Omega \leftrightarrow \nabla \cdot \vec{E} = \dfrac{\rho_\Omega}{\epsilon} \)
\end{itemize}

\subsection{Campo Elettrico}
\begin{itemize}
	\item Campo \(\vec{E}\) in presenza di cariche puntiformi \( \vec{E} = \dfrac{Q}{4 \pi \epsilon_0} \cdot \dfrac{(\vec{R_q} - \vec{R_p})}{| \vec{R_q} - \vec{R_p} | ^3 } = \dfrac{Q}{4 \pi \epsilon_0} \cdot \dfrac{1}{R^2} \vec{a_r} \)
	\item Densità di flusso elettrico \( \vec{D} = \epsilon_0 \vec{E} \),  \( \vec{E} = \dfrac{Q}{4 \pi} \cdot \dfrac{(\vec{R_q} - \vec{R_p})}{| \vec{R_q} - \vec{R_p} | ^3 } = \dfrac{Q}{4 \pi} \cdot \dfrac{1}{R^2} \vec{a_r} \)
	\item Densità superficiale di carica \( \sigma = D_n = \epsilon E_n \)6t
	\item Momento di dipolo elettrico \( \vec{p} = Q \vec{d} \)
	\item Potenziale elettrostatico \(\displaystyle dV = -\vec{E} \cdot \vec{dl} \Rightarrow V = -\int{\vec{E} \cdot \vec{dl}} \)
	\item Relazioni tra \(\vec{D}\) ed \(\vec{e}\) \( \nabla \cdot (\epsilon_0 \vec{E} + \vec{P} ) = \rho_\Omega \), \( \vec{D} = \epsilon_0 \vec{E} + \vec{P} \)
	\item All'interno di mezzi lineari \(\vec{P} = \epsilon_0 \chi \vec{E} \) con \(\chi\) detta suscettività elettrica, \(\chi \geq  0\)
	\item Relazione tra \(\epsilon\) e \(\chi\) \(\epsilon = (1 + \chi_e) \epsilon_0 = \epsilon_0 \epsilon_r \rightarrow \epsilon_r = 1 + \chi_e \rightarrow \vec{D} = \epsilon \vec{E} = \epsilon_0 \epsilon_r \vec{E} = \epsilon_0 (1 + \chi_e) \vec{E} = \epsilon_0 \vec{E} + \vec{P} \)
	\item Relazione tra \(\vec{P}\), \(\vec{D}\),\(\vec{E}\) nei mezzi isotropi \(\vec{P}\|\vec{D}\|\vec{E}\)
	\item Energia del sistema \(\displaystyle W_e = \dfrac{1}{2} \sum\limits_{i=1}^{n}Q_iV_i = \dfrac{1}{2} \int\limits_\Omega \rho_\Omega V d\Omega = \dfrac{1}{2} \int\limits_{s} \vec{D} \cdot \vec{E} d\Omega = \dfrac{1}{2} \int\limits_{s} \epsilon |\vec{E}|^2 d\Omega \)
	\item Densità di energia \( w_e = \dfrac{1}{2} \vec{D} \cdot \vec{E} = \dfrac{1}{2} \epsilon |\vec{E}|^2 \)
\end{itemize}

\subsubsection{Interfaccia tra due mezzi}
\begin{itemize}
	\item \textbf{conduttori} (componente tangenziale) \(E_{1t} = E_{2t}\), \(\vec{a_n} \times (\vec{E_2} - \vec{E_1}) = 0 \), \( \dfrac{\vec{D_{1t}}}{\vec{D_{2t}}} = \dfrac{\epsilon_1}{\epsilon_2} \)
	\item \textbf{conduttori} (componente normale) \( D_{2n} - D_{1n} = \rho_s \), \(\vec{a_n} \cdot (\vec{D_2} - \vec{D_1}) = \rho_s \), \(\epsilon_{2n} E_{2n} - \epsilon_{1n} E_{1n} = \rho_s \)
	\item \textbf{dielettrici} (componente tangenziale) \( E_{2t} = E_{1t} \), \( D_{2t} = \dfrac{\epsilon_2}{\epsilon_1} D_{1t} \)
	\item \textbf{dielettrici} (componente normale) \( D_{2n} = D_{1n} \), \( E_{2n} = \dfrac{\epsilon_1}{\epsilon_2} E_{1n} \)
	\item \textbf{conduttore e dielettrico} (componente tangenziale) \( E_{2t} = 0\), \(D_{2t} = 0\)
	\item \textbf{conduttore e dielettrico} (componente normale) \( D_{2n} = \rho_s \), \( E_{2n} = \dfrac{\rho_s}{\epsilon_2} \)
\end{itemize}

\subsection{Capacità elettrica}
\begin{itemize}
	\item Condensatore a facce piane parallele \( C = \dfrac{Q}{V} = \dfrac{\epsilon_0 \epsilon_r A}{d} \)
	\item Formula generale \(\displaystyle C = \dfrac{Q}{V} = \dfrac{\oint\limits_{S}\vec{D} \cdot \vec{dS}}{-\int\limits_{P_1}^{P_2} \vec{E} \cdot \vec{dl}} \)
	\item Energia immagazzinata in un condensatore \(W_e = \dfrac{1}{2} Q V = \dfrac{1}{2} C V^2 = \dfrac{1}{2} \dfrac{Q^2}{C} \)
\end{itemize}

\subsection{Corrente elettrica / Legge di Ohm}
\begin{itemize}
	\item Velocità di deriva \(\vec{v_d} = \mu_q \vec{E} \), \( \vec{v_d} \| \vec{E} \)
	\item Densità di corrente \( \vec{J} = qN \vec{v_d} = qN \mu_q \vec{E} = \sigma \vec{E} \) con \(\sigma\) detta conducibilità del mezzo
	\item In un conduttore ideale, si ha \(\mu \rightarrow \infty\), \(\sigma \rightarrow \infty \)
\end{itemize}

\newpage

\section{Magnetostatica}
\begin{itemize}
	\item Legge di Ampere \( \displaystyle \oint\limits_{c} \vec{H} \cdot \vec{dl} = \int\limits_{s} \vec{J} \cdot \vec{ds} \leftrightarrow \nabla \times \vec{H} = \vec{J} \)
\end{itemize}

\subsection{Campo Magnetico}
\begin{itemize}
	\item Legge di Biot Savart (differenziale) \( \vec{dF_{12}} = \dfrac{\mu_0}{4 \pi} \dfrac{I_1 \vec{dl_1} \times [I_2 \vec{dl_2} \times \vec{a_{12}}] }{R^2} \)
	\item Legge di Biot Savart (integrale) \(\displaystyle \vec{F_{12}} = \dfrac{\mu_0 I_1 I_2}{4 \pi} \oint\limits_{c_1} \oint\limits_{c_2} \dfrac{\vec{dl_1} \times [\vec{dl_2} \times \vec{a_{12}}]} {R^2} = \dfrac{\mu_0 I_1 I_2}{4 \pi} \oint\limits_{c_1} \oint\limits_{c_2} \dfrac{\vec{a_{12}} \cdot [\vec{dl} \cdot \vec{d2}]} {R^2} \), \(\vec{F_{12}} = - \vec{F{21}} \)
	\item Densità di flusso magnetico (differenziale) \( \vec{dB} = \dfrac{\mu_0}{4 \pi} \dfrac{I_2 \vec{dl_2} \times \vec{a_{12}}}{R^2} \)
	\item Densità di flusso magnetico (integrale) \(\displaystyle \vec{B} = \oint\limits_{c_2} \dfrac{\mu_0}{4 \pi} \dfrac{I_2 \vec{dl_2} \times \vec{a_{12}}}{R^2} = \oint\limits_{c_2} \dfrac{\mu_0}{4 \pi} \dfrac{\vec{J} \cdot ( \vec{dl_{1}} \cdot \vec{dl_2} )}{R^2}\)
	\item Campo magnetico \( \vec{H} = \dfrac{\vec{B}}{\mu_0} \)
	\item Autoinduttanza magnetica \(\displaystyle L_{11} = \dfrac{\Phi{11}}{I_1} = \dfrac{\int_{S_1} \vec{B_1} \cdot \vec{dd_1}}{I_1} \) con \( \vec{B_1} = \dfrac{\mu I_1}{4 \pi} \oint\limits_{c1} \dfrac{\vec{dl} \cdot \vec{ar}}{R^2} \), \(\displaystyle \Phi_{m, 11} = \dfrac{\mu I_1}{4 \pi} \int\limits_{s_1} ( \oint\limits_{c1} \dfrac{\vec{dl} \cdot \vec{a_r}}{R^2}) ds \linebreak \Rightarrow L_{11} = \dfrac{\mu}{4 \pi} \int\limits_{s_1} ( \oint\limits_{c_1} \dfrac{\vec{dl} \times \vec{a_r}}{R^2}) \cdot \vec{ds}\)
	\item Mutua induttanza \(\displaystyle L_{21} = \dfrac{\mu}{4 \pi} \int\limits_{s_2} \left( \int\limits_{c_1} \dfrac{\vec{dl} \times \vec{a_r}}{R^2} \right) \cdot \vec{ds} \)
	\item Energia del sistema \( W_m = \dfrac{1}{2} L_{11} I_1^2 \)
	\item Densità di energia  \( w_m = \dfrac{1}{2} \vec{B} \cdot \vec{H} = \dfrac{1}{2} \mu H^2 \)
\end{itemize}

\subsubsection{Campo Magnetico nei materiali}
\begin{itemize}
	\item Momento di dipolo magnetico \( \vec{m} = A \cdot I \cdot \vec{a_n} \)
	\item Densità di momento magnetico \(\vec{M} = \lim\limits_{\Delta\Omega \rightarrow 0} \dfrac{\sum_i \vec{m_i}}{\Delta\Omega}\) se il mezzo è lineare \( \vec{M} = \chi_m \vec{H} \) con \(\chi_m\) detta suscettività magnetica
	\item Permeabilità del mezzo \( \mu = \mu_0 \mu_r = \mu_0 (1 + \chi_m) \)
	\item Relazione tra \( \vec{H} \) e \( \vec{M} \): \( \vec{B} = \mu \vec{H} = \mu_0 \mu_r \vec{H} = \mu_0 (1 + \chi_m) \vec{H} = \mu_0 (\vec{H} + \vec{M}) \)
\end{itemize}



\subsubsection{Interfaccia tra due mezzi}
\begin{itemize}
	\item Componente tangenziale \( H_{2t} - H_{1t} = J_{sn} \), \( \dfrac{J_{2t}}{\sigma_2} = \dfrac{J_{1t}}{\sigma_1} \)
	\item Componente normale \( H_{2n} = \dfrac{\mu_1}{\mu_2} H_{1n} \), \(B_{2n} = B_{1n}\), \( \vec{a_n} \cdot \left(\vec{B_2} - \vec{B_1} \right) = 0 \), \( \rho_s = \left( \epsilon_2 - \epsilon_1 \dfrac{\sigma_2}{\sigma_1} \right) E_{2n} \),  \( \rho_s = \left( \epsilon_1 - \epsilon_2 \dfrac{\sigma_1}{\sigma_2} \right) E_{1n} \)
\end{itemize}

\subsection{Resistenza elettrica / Legge di Joule}
\begin{itemize}
	\item Resistenza \(\displaystyle  R = \dfrac{V}{I} = \dfrac{\displaystyle-\int_{p_1}^{p_2} \vec{E} \cdot \vec{dl}}{\displaystyle\oint_s \vec{J} \cdot \vec{ds}} = \dfrac{1}{\sigma} \dfrac{\displaystyle\int_{p_1}^{p_2} \vec{E} \cdot \vec{dl}}{\displaystyle \oint_s \vec{E} \cdot \vec{ds}} \)
	\item Relazione tra resitenza e capacità \(R \cdot C = \dfrac{\epsilon}{\sigma} \leftrightarrow \dfrac{G}{C} = \dfrac{\sigma}{\epsilon} \)
	\item Legge di Joule (differenziale) \( \dfrac{\partial P}{\partial \Omega} = \vec{E} \cdot \vec{J} \) detta anche potenza specifica
	\item Legge di Joule (integrale) \(\displaystyle P = \int\limits_\Omega \vec{E} \cdot \vec{J} \; d\Omega \)
\end{itemize}

\newpage

\section{Regime dinamico}
\begin{itemize}
	\item Legge di Faraday \(\displaystyle \oint\limits_c \vec{E} \cdot \vec{dl} = - \int\limits_s \dfrac{\partial \vec{B}}{\partial t} \cdot \vec{ds} \) se la superficie non cambia nel tempo
	\item Circuitazione di \( \vec{H} \) e corrente di spostamento \(\displaystyle \oint\limits_c \vec{H} \cdot \vec{dl} = \int\limits_s \vec{J} \cdot \vec{ds} + \dfrac{d}{dt} \int\limits_s \vec{D} \cdot \vec{ds} \)
	\item Legge di conservazione della carica \(\displaystyle \oint\limits_s \vec{J} \cdot \vec{ds} = -\dfrac{d}{dt} \int\limits_\Omega \rho_\Omega \, d\Omega \)
	\end{itemize}
	
\subsection{Teorema di Poyinting}
\begin{itemize}
	\item Vettore di Poynting \( \vec{S} = \vec{E} \times \vec{H} \)
	\item Vettore di Poyting (dominio dei fasori) \( \vec{S} = \dfrac{1}{2} \vec{E} \times \vec{H}^{*} \)
	\item Teorema di Poynting \( P_{diss} = \oint\limits_\Sigma \vec{S} \cdot d \,\Sigma \)
	\item Vettore di Poynting associato ad un onda piana \( \vec{S_{ist}} = \dfrac{A^2}{\eta_0} \cos^2{(\omega t)} \, \vec{a_z} \), \( \vec{S_{ave}} = \dfrac{A^2}{2 \eta_0} \, \vec{a_z}\)
	\item Vettore di Poynting associato ad un onda piana (dominio dei fasori) \( \vec{S_{ist}} = \dfrac{1}{2} Re[\vec{E} \times \vec{H}^{*}] \)
\end{itemize}

\section{Onde Piane}
\begin{itemize}
	\item Equazione di Helmoltz (onda piana uniforme senza perdite) \( \dfrac{\partial^2 E(z, t)}{\partial t^2} - \mu \epsilon \dfrac{\partial^2 E(z, t)}{\partial z^2} = 0\)
	\item Equazione di Helmoltz (dominio dei fasori) \( \nabla^2 \vec{E} = \gamma^2 \vec{E} \)
	\item Lunghezza d'onda nel vuoto \( \lambda = \dfrac{2 \pi}{\omega} = \dfrac{c}{f} \)	
	\item Lunghezza d'onda nel mezzo \( \lambda = \dfrac{c}{f} \dfrac{1}{\sqrt{\mu \epsilon}} \)
	\item Costante di propagazione \( \gamma = \sqrt{j\omega\mu (\sigma + j \omega \epsilon) } = \alpha + j \beta \) dove \( \alpha = costante\ di\ attenuazione > 0\), \newline \( \beta = costante\ di\ fase = \dfrac{2 \pi}{\lambda} \)
	\item Velocità della luce (velocità di propagazione delle onde) \( c = \dfrac{1}{\sqrt{\mu \epsilon}} \cong 3 \cdot 10^8 \)
	\item Impedenza intrinseca del mezzo \( \eta = \dfrac{E^+}{H^+} \)
	\item Impedenza d'onda \( Z = \dfrac{E^+ - E^-}{H^+ - H^-} \)
	\item Indice di rifrazione \( n = \dfrac{c}{v} = \sqrt{\epsilon_r \mu_r} \)
	\item Densità di potenza trasportata \( |S| = \dfrac{1}{2} \dfrac{|\vec{E}|^2}{\eta} = \dfrac{1}{2} |\vec{H}|^2 \eta \)
\end{itemize}

\subsection{Polarizzazione}
\begin{itemize}
	\item Sia \( \vec{E}(x, y, z. t) \) un campo elettrico con componenti in sole \( x \) e \( z \). Allora, sul piano trasverso ( \( z = 0 \) ) si ottiene:
	\begin{itemize}
		\item \( \vec{E}(z, t) = E_x \cos ( \omega t) \vec{a_x} + E_Y \cos( \omega t + \phi_0) \vec{a_y}  \)
		\item Si distinguono due casi particolari:
		\begin{enumerate}
			\item \( \phi_0 = 0 \), \( E_x, E_y \) qualsiasi. Allora: \( \xi  = \arctan\left(\dfrac{E_y}{E_x}\right) \), \( |\vec{E}(0, t)|^2 = (E_x^2 + E_y^2) \cos(\omega t) \) \textbf{Polarizzazione lineare}
			\item  \( \phi_0 = \pm \dfrac{\pi}{2} \), \( E_x = E_y = E \). Allora: \( \xi(t)  = \mp \omega t \), \( |\vec{E}|^2 = E^2 \) \textbf{Polarizzazione circolare}
		\end{enumerate}
	\end{itemize}	  
\end{itemize}

\subsection{Incidenza delle onde}

\subsubsection{Incidenza normale su discontinuità piana}
Mezzi ideali e senza perdite, onda elettromagnetica con componenti in \( x \) e \( y \) nella sezione \( z = cost \)
\begin{itemize}
	\item Coefficiente di riflessione \( \Gamma = \Gamma(0) \exp(2 j \beta z)  \), dove \( \Gamma(0) = \dfrac{n_2 - n_1}{n_2 + n_1}\), \( | \Gamma(0) | \leq 1 \)
	\item Coefficiente di trasmissione \( T = T(0) \exp(2 j \beta z)  \), dove \( T(0) = \dfrac{2 n_2}{n_2 + n_1} = 1 + \Gamma(0)\), \( | T(0) | \leq 2 \)
	\item Onda riflessa \( E_1^-(0) =  E_1^+(0) \cdot \Gamma (0) \)
	\item Onda trasmessa \( E_2^+(0) =  E_1^+(0) \cdot T (0) \)
	\item Impedenza d'onda \( Z(z) = \eta_1 \left(\dfrac{1 + \Gamma(z)}{1 - \Gamma(z)}\right) \)
\end{itemize}

\subsubsection{Incidenza non normale}
\textbf{Ipotesi:} onda su piano \(xz\)
\begin{itemize}
	\item Impedenza TE \( \eta^{TE}_n = \dfrac{\eta}{cos(\theta_n)} \)
	\item Impedenza TM \( \eta^{TM}_n = \eta \cdot cos(\theta_n) \)
	\item Componente TE dell'onda ha componente \(y\)
	\item Componente TE dell'onda ha componenti \(xz\)
	\item Angolo di incidenza \( \theta = \arctan\left( \dfrac{\beta_z}{\beta_x} \right) \)
	\item Costante di propagazione \(\gamma \rightarrow\) va proiettata nelle direzioni \(x\) e \(y\) tramite \(\sin\) e \(\cos\)
\end{itemize}

\subsubsection{Trasmissione totale}
\begin{itemize}
	\item Indice di rifrazione \( \Gamma = 0 \leftrightarrow Z_L = Z_{in} \)
	\item Angolo di Brewster \( \theta_P = \arcsin\left(\sqrt{\dfrac{\epsilon_2}{\epsilon_1 + \epsilon_2}}\right) = \arctan\dfrac{n_2}{n_1}\)
\end{itemize}

\subsection{Mezzi attraversati dalle onde}
\subsubsection{Mezzo senza perdite}
\begin{itemize}
	\item \( \sigma = 0 \Rightarrow \gamma = j \omega \sqrt{\mu \epsilon} \Rightarrow \alpha = 0, \beta = \omega \sqrt{\mu \epsilon} \) 
	\item \( \eta = \sqrt{\dfrac{\mu}{\epsilon}}  = 377 \, \Omega \)
	\item \( v = \dfrac{1}{\sqrt{\mu \epsilon}} = c \cong 3\times10^8 \, m/s \)
	\item \( \lambda = \dfrac{v}{f} \)
	\item Impedenza intrinseca del vuoto (dominio dei fasori) \( \dfrac{\vec{E^+}}{\vec{H^+}} = \dfrac{j \omega \mu}{\gamma} = \eta \), \(\dfrac{\vec{E^-}}{\vec{H^-}} = - \dfrac{j \omega \mu}{\gamma} = - \eta\) 
\end{itemize}

\subsubsection{Buon conduttore}
\begin{itemize}
	\item \( \sigma >> \omega \epsilon \Rightarrow \gamma = \sqrt{-\omega^2 \mu \epsilon} \)
	\item \( \eta = \dfrac{1+j}{sqrt{2}} \sqrt{\dfrac{\pi f \mu}{\sigma}} \Rightarrow \alpha \cong \beta
	\cong \sqrt{\dfrac{\omega \mu \sigma}{2}} \)
	\item \( v \cong \dfrac{\omega}{\beta} \cong \sqrt{\dfrac{2 \omega}{\mu \sigma}} \)
	\item \( \lambda = 2 \pi \delta = \dfrac{v}{f} \)
	\item Spessore pelle \( \delta = \dfrac{1}{\alpha} = \dfrac{1}{\sqrt{\pi f \mu \sigma}} \)
	\item Costante dielettrica \( \epsilon = \epsilon ^ { ' } + j \epsilon ^ { '' } \)
	\item Costante di permeabilità magnetica \( \mu = \mu ^ { ' } + j \mu ^ { '' } \)
\end{itemize}

\newpage

\section{Linee TEM}
\begin{itemize}
	\item Equazione 1 \( \dfrac{\partial V(z, t)}{\partial z} = - \dfrac{\partial I(z, t)}{\partial z} \cdot L\) con \(L = \) induttanza per unità di lunghezza
	\item Equazione 2 \( \dfrac{\partial I(z, t)}{\partial z} = - \dfrac{\partial V(z, t)}{\partial z} \cdot C\) con \(C = \) capacità per unità di lunghezza
	\item Uguaglianze \( \dfrac{G}{C} = \dfrac{\sigma}{\epsilon} \), \(L_0 C_0 = \mu_0 \epsilon_0  \)
	\item Impedenza \( Z = \sqrt{\dfrac{L}{C}} \)
	\item Velocità di fase nel conduttore \( v = \dfrac{1}{\sqrt{LC}} \)
\end{itemize}

\subsection{Potenza ed energia in una linea}
\begin{itemize}
	\item Potenza disponibile \( P_D = \dfrac{|V_g|^2}{8 R_g} = P_D \)
	\item Potenza sul carico \( P_L = P_D ( 1 - | \Gamma_L | ^ 2 ) = \dfrac{|V_g|^2}{8 R_g} P_D ( 1 - | \Gamma_L | ^ 2 ) \) 
	\item Potenza sul carico (in funzione della tensione) \( P_L = \dfrac{1}{2} |V|^2  \operatorname{Re}(Y) \)
	\item Potenza sul carico (in funzione della corrente) \( P_L = \dfrac{1}{2} |I|^2  \operatorname{Re}(Z) \)
	\item Densità di energia trasmessa \( S_{tra} = S_{inc} \left( 1 - \left|\Gamma \right| ^ 2 \right) \) 
\end{itemize}

\subsection{Corrente e tensione in una linea non attenuativa}
\begin{itemize}
	\item Tensione \(|V_L| = \left| V^+(0) \right| \left|1 + \Gamma_L \right|\)
	\item Corrente \(|I_L| = \left| I^+(0) \right| \left|1 - \Gamma_L \right|\)
	\item Nel caso di un corto circuito - \textit{come negli stub c.c.}:
	\begin{itemize}
		\item Tensione \( |V(d)| = |2 V(0)| \sin(\beta d) | \) - il suo massimo si troverà a \( \dfrac{\lambda}{4} \) dal \textit{c.c.}
		\item Corrente \( |I(d)| = \dfrac{|V_g|}{Z_C} |\cos (\beta d) |\)
	\end{itemize}
\end{itemize}

\subsection{Cavo coassiale}
\textbf{Ipotesi:} \(a = \) raggio interno, \(b = \) raggio esterno 
\begin{itemize}
	\item Capacità \( C = \dfrac{2 \pi \epsilon}{\ln\left(\dfrac{b}{a}\right)} \) con \( \epsilon = \epsilon_0 \epsilon_r \)
	\item Induttanza \( L = \dfrac{\mu_0}{2 \pi} \ln\left(\dfrac{b}{a}\right) \)
	\item Attenuazione conduttore \( \alpha_c = \dfrac{R}{2 Z_C} \dfrac{Np}{m}\)
	\item Attenuazione dielettrico \( \alpha_d = \dfrac{G Z_C}{2} = \dfrac{\pi}{\lambda}\dfrac{\epsilon"}{\epsilon'} \dfrac{Np}{m}\)
	\item Impedenza \( Z_C = \sqrt{\dfrac{L}{C}} = \dfrac{\eta}{2 \pi} \ln\left(\dfrac{b}{a} \right) =  \dfrac{1}{2 \pi} \sqrt{\dfrac{\mu}{\epsilon}} \ln\left(\dfrac{b}{a}\right)\)
	\item Resistenza superficiale \( R_s = \sqrt{\dfrac{\pi f \mu}{\sigma}} = \sqrt{\dfrac{\omega \mu}{2 \sigma}} = \dfrac{1}{\sigma \delta} \)
	\item Resistenza per unità di lunghezza \( R = \dfrac{R_s}{2 \pi} \left( \dfrac{1}{a} + \dfrac{1}{b} \right) \)
	\item Conduttanza per unità di lunghezza \( G = C \dfrac{\omega\epsilon"}{\epsilon'} = \dfrac{2 \pi \omega \epsilon"}{\ln\left(\dfrac{b}{a} \right)} \)
\end{itemize}

\subsection{Linea a striscia}
\begin{itemize}
	\item Capacità \( C = \epsilon_0 \epsilon_r \dfrac{w}{h} \)
	\item Induttanza \( L = \mu_0 \dfrac{h}{w} \)
	\item Ammettenza per unità di lunghezza \( g = \sigma_d \dfrac{\textnormal{Area}}{h}\)
	\item Attenuazione dovuta al dielettrico \( \alpha_d = \dfrac{g}{2 Y_c} \)
\end{itemize}

\section{Linee quasi TEM}
\begin{itemize}
	\item Costante dielettrica efficace \( \epsilon_{eff} = \dfrac{LC}{\mu_0} \)
\end{itemize}

\newpage 

\section{Linee TE in guida rettangolare}
Sia \(a\) il lato della guida che giace sull'asse \(x\) e sia \(b\) il lato della guida che giace sull'asse \(y\). Allora le ampiezze e le frequenze di taglio dei modi \(TM_{mn} \) sono:
\vspace{15pt}
\begin{center}
	\begin{tabular}{|Sc|Sc|Sc|}
		\hline 
		Modo & Lunghezza di taglio & Frequenza di taglio \\ 
		\hline 	
		\(TE_{10}\) & \(\lambda_c = 2a\) & \(f_c = \dfrac{c}{2a}\) \\ 
		\hline 
		\(TE_{01}\) & \(\lambda_c = 2b\) & \(f_c = \dfrac{c}{2b}\) \\ 
		\hline 
		\(TE_{20}\) & \(\lambda_c = a\) & \(f_c = \dfrac{c}{a}\) \\ 
		\hline 
		\(TE_{02}\) & \(\lambda_c = b\) & \(f_c = \dfrac{c}{b}\) \\ 
		\hline 
	\end{tabular} 
\end{center}
\vspace{5pt}

\begin{itemize}
	\item Pulsazione di taglio \( \omega_c = \dfrac{1}{\sqrt{\omega \epsilon}} \sqrt{\left( \dfrac{m \pi}{a}\right)^2 + \left( \dfrac{m \pi}{b} \right)^2 }\) con \(a\), \(b\) interi
	\item Impedenza modale \( Z_{te} = \dfrac{\eta}{\sqrt{1-\left(\dfrac{\omega_c}{\omega} \right) }} = \dfrac{\eta}{\sqrt{1-\left(\dfrac{f_c}{f} \right) }} \)
	\item Frequenza di taglio \( f_c = \dfrac{1}{2a \sqrt{\mu \epsilon}} \)
	\item Velocità di gruppo \( v_g = v \sqrt{1- \left( \dfrac{\omega_c}{\omega} ^ 2 \right)} \)
	\item Lunghezza d'onda di gruppo \( \lambda_g = \dfrac{\lambda}{\sqrt{1- \left( \dfrac{\omega_c}{\omega} ^ 2 \right)}}\)
	\item Potenza trasportata \( P = \dfrac{|E_0|^2 a b}{4 \cdot Z_{te}} \)
	\item Coefficiente di attenuazione \( \alpha = \dfrac{2 \pi}{\lambda_c} \sqrt{1 - \left( \dfrac{f}{f_c} \right) ^ 2} \)
\end{itemize}


\newpage

\section{Adattamento di impedenza}
L'obiettivo dell'adattamento di impedenza è portare la massima potenza disponibile sul carico \(P_L = P_D\) annullando quindi il coefficiente di riflessione \( (\Gamma = 0) \).

\subsection{Strutture adattanti}
Esistono 3 tipologie di strutture adattanti:\
\begin{itemize}
	\item Trasformatore \( \dfrac{\lambda}{4} \) - solo per carichi \textbf{reali}
	\item Stub semplice - per carichi \textbf{reali} o \textbf{complessi}
	\item Stub doppio - per carichi \textbf{reali} o \textbf{complessi}
\end{itemize}

\subsubsection{Trasformatore lambda-quarti}
Funziona esclusivamente con carichi reali. \`E costituito da un pezzo di conduttore lungo un quarto della lunghezza d'onda \( \lambda \).
Impedenza del trasformatore: \(Z_x = \sqrt{Z_{in} \cdot Z_L} \) con \(Z_{in}\) impedenza di ingresso e \(Z_L\) impedenza di carico.

\subsubsection{Trasformatore lambda-quarti con neutralizzazione}
Risolve il problema dell'impossibilità dei trasformatori lambda-quarti di adattare carichi complessi.
\`E composto da un tratto di neutralizzazione lungo \(l_n\) e da un trasformatore lambda-quarti di impedenza \(Z_x\).
Il tratto di neutralizzazione sarà necessario a trasformare in impedenza puramente reale il carico.
Il trasformatore lambda-quarti adatterà l'impedenza al valore caratteristico.

Operativamente, si dovrà:
\begin{enumerate}
	\item Normalizzare l'impedenza del carico \(Z_L\) all'impedenza caratteristica \(Z_C\) ottendendo \(\overline{Z_L}\)
	\item Tracciare la circonferenza con centro in \(1\) e passante per \( \overline{Z_L} \)
	\item Partendo da \(\overline{Z_L}\), ruotare in senso orario sulla circonferenza appena tracciata fino ad intersecare l'asse reale nel punto \(A\)
	\item Il valore di \( l_n \), normalizzato alla lunghezza d'onda \( \lambda \), sarà letto come differenza tra il prolungamento del punto \(A\) sulla scala esterna della carta di Smith e medesimo prolungamento di \(\overline{Z_L}\)
	\item Denormalizzare \(\overline{Z_L}\) per ottenere \(Z_L\)
	\item Il valore dell'impedenza del trasformatore sarà data da \(Z_x = \sqrt{Z_{in} \cdot Z_L} \)
\end{enumerate}

\subsubsection{Stub semplice}
Detto anche stub \textit{singolo}, è realizzato con un tratto di linea di trasmissione in \textit{c.c}. o in \textit{c.a} di lunghezza \(l_s\), opportunamente collegato in serie o in parallelo alla linea ad un tratto di distanza \(d_s\) dal carico.
Esistono due tipi di stub semplice:
\begin{enumerate}
	\item Stub parallelo - si lavora con le ammettenze \( \overline{Y_L} = \dfrac{Y_L}{Y_C} = \dfrac{Z_C}{Z_L} \) 
	\item Stub serie - si lavora con le impedenze \( \overline{Z_L} = \dfrac{Z_L}{Z_C} \)
\end{enumerate}

Si può cercare \(l_s\) in modo che dia origine ad un \textit{corto circuito} o a un \textit{circuito aperto}, mentre \(d_s\) potrà assumere un solo valore.
\begin{enumerate}
	\item Il \textit{circuito aperto} si troverà a destra dell'asse reale della carta di Smith
	\item Il \textit{corto circuito} si troverà a sinistra dell'asse reale della carta di Smith
\end{enumerate}
La differenza tra stub in \textit{circuito aperto} e in \textit{corto circuito} sarà pari a mezzo giro sulla carta di Smith \(( l = \dfrac{\lambda}{4} )\). 

Operativamente, per gli stub \textit{serie} si dovrà:
\begin{enumerate}
	\item Normalizzare l'impedenza del carico \(Z_L\) all'impedenza caratteristica \(Z_C\) (o impedenza di ingresso) ottendendo \(\overline{Z_L}\)
	\item Segnare sulla carta di Smith il valore di \( \overline{Z_L} \)
	\item Tracciare la circonferenza a modulo costante pari a \( \overline{Z_L} \)
	\item Partendo da \(\overline{Z_L}\), ruotare in senso orario sulla circonferenza appena tracciata fino ad intersecare la circonferenza di raggio \(\dfrac{R_g}{Z_C}\) nel generico punto \(A\)
	\item Il valore di \( d_s \), normalizzato alla lunghezza d'onda \( \lambda \), sarà letto come differenza tra il prolungamento del punto \(A\) sulla scala esterna della carta di Smith e medesimo prolungamento di \(\overline{Z_L}\)
	\item Partendo da \(A\) si procede ruotando fino a \(Z = 0\) (per uno stub c.c.) o \( Z = \inf \) (per uno stub c.a.) nel generico punto \(B\)
	\item Analogamente a quanto trovato per \( d_s \), la lunghezza dello stub sarà pari alla differenza tra il prolungamento del punto \(B\) e medesimo prolungamento di \(A\)  
\end{enumerate}
Il procedimento sarà analogo per gli stub \textit{parallelo} ma si dovrà lavorare con le ammettenze al posto delle impedenze.

\subsubsection{Stub doppio}
\`E una struttura adattante formata da due stub semplici di lunghezza \(l_1\) e \(l_2\) posti a distanza \(d_s\) (fissata) tra di loro. I due stub possono essere sia collegati in serie che in parallelo
\begin{enumerate}
	\item La lunghezza del primo stub (il più vicino al carico) è ricercata in modo da eguagliare la parte reale dell'impedenza carico a quella della linea
	\item La lunghezza del secondo stub è ricercata in modo da annullare la parte immaginaria dell'impedenza di carico
\end{enumerate}

Operativamente, per gli stub \textit{parallelo} si dovrà
\begin{enumerate}
	\item Normalizzare l'ammettenza del carico \(Y_L\) all'impedenza caratteristica \(Y_C\) (o impedenza di ingresso) ottendendo \(\overline{Y_L}\)
	\item Segnare sulla carta di Smith il valore di \(\overline{Y_L}\)
	\item Tracciare la circonferenza a parte reale costante pari a \(\operatorname{Re}(\overline{Y_L}) \) \textit{"circonferenza di partenza"}
	\item Rutotare la circonferenza di raggio \(\dfrac{R_g}{Z_C}\) in senso antiorario di un angolo pari a \(d_s\) attorno al suo centro \textit{"circonferenza di arrivo"}
	\item Si trovano quindi 2 intersezioni tra la due circonferenze ed è necessario sceglierne uno \textit{"punto di partenza" \(A\)}
	\item Tracciare la circonferenza con centro in \(1\) e passante per \(A\) e ruotare di una lunghezza pari a \(d_s\) fino ad arrivare al \textit{"punto di arrivo" \(B\)}
	\item Calcolare le ammettenze dei due stub \(Y_{S1} = A - \overline{Y_L}\) e \(Y_{S2} = -\operatorname{Im}(B) \)
	\item Le lunghezze dei due stub saranno quelle che portano i loro rispettivi valori di ammettenza tale da avere un c.c. \( Y = \inf \) o un c.a. \( Y = 0 \), dipendentemente dalla struttura che si sta cercando di realizzare
\end{enumerate}
Il procedimento sarà analogo per gli stub \textit{serie} ma si dovrà lavorare con le impedenze al posto delle ammettenze.
\textbf{Nota:} generalmente \(d_s\) è un dato del problema.

\subsection{Linea attenuativa}
Per calcolare l'impedenza ad una certa distanza \(l\) dal carico di una linea attenuativa, bisogna
\begin{enumerate}
	\item Assicurarsi che il valore di \(\alpha\) sia espresso in \(\dfrac{N_p}{m}\)
	\item Normalizzare l'impedenza del carico \(Z_L\) all'impedenza caratteristica \(Z_C\) (o impedenza di ingresso) ottendendo \(\overline{Z_L}\)
	\item Tracciare la circonferenza con centro in \(1\) e passante per \( \overline{Z_L} \)
	\item Partendo da \(\overline{Z_L}\), ruotare in senso orario sulla circonferenza appena tracciata fino a percorrere una lunghezza normalizzata alla lunghezza d'onda pari alla lunghezza della linea
	\item Tracciare un segmento che congiunga il centro della carta di Smith con il punto appena trovato
	\item Calcolare il modulo del coefficiente di riflessione \(\Gamma\) in corrispondenza del carico e riportarlo in corrispondenza del generatore moltiplicandolo per un fattore \(\;e^{-2 \alpha l}\)
	\item Misurare sulla scala lineare più esterna della carta di Smith (con la dicitura \textit{TRASM. COEFF.}) una lunghezza \(l_\alpha\) pari al valore appena trovato
	\item L'impedenza attenuata dalla linea (normalizzata) \(\overline{Z_{L_{att}}}\) sarà trovata sul segmento prima tracciato, a distanza \( l_\alpha\) dal centro
	\item Denormalizzare il valore appena trovato per calcolare il valore effettivo di \( Z_{L_{att}} \)
	\item Calcolare il nuovo coefficiente di riflessione \(\Gamma\) tra impedenza del generatore \(Z_g\) e impedenza attenuata del carico \(Z_{L_{att}}\)
		
\end{enumerate}

\subsection{Potenza in una linea adattata}
\begin{itemize}
	\item Potenza al carico \(P_L = P_D = \dfrac{|V_g|^2}{8 R_g} \)
	\item La potenza disponibile è uguale in qualsiasi punto della linea
\end{itemize}

\subsection{Tensione in uno stub c.c}
La tensione avrà un massimo per \( l = \dfrac{\lambda}{4} \) e sarà nulla in \(l = 0\) (considerando un asse parallelo a \(l\) e nullo nel punto di c.c.). \\\
Il massimo varrà \( V_{max} = \dfrac{|V(l_s)|}{sin(\beta l_s)} \)

\subsection{Note sulla Carta di Smith}
\begin{itemize}
	\item Ruotare in senso orario corrisponde ad una direzione verso il carico (allontanandosi quindi dal generatore)
	\item 1 giro completo della carta corrisponde a \(0.5 \lambda\)
	\item La carta di Smith può essere usata indifferentemente con impedenze e ammettenze
	\item Ogni tacca sulla circonferenza esterna corrisponde a \( \dfrac{1}{500} = 0.002 \) di lambda
\end{itemize}

\newpage

\section{Antenne}
\begin{itemize}
	\item Direttività \( \displaystyle D = \dfrac{4 \pi}{\int{f(\theta, \phi) d \Omega}} = \dfrac{S_{max}}{S_{iso}} \)	
	\item Area efficace \( \displaystyle A_e = \dfrac{|l_e| ^ 2 \eta_0}{4 R} \)
	\item Relazione universale \( \dfrac{G}{A_e} = \dfrac{4 \pi}{\lambda^2} \)
	\item Tensione a vuoto \( V_0 = l_e \cdot E_{inc} \cdot \sqrt{f_r(\theta, \phi} \)
	\item Potenza ricevuta \(P_R = P_D = S_{inc} \cdot A_e \cdot  f(\theta, \phi) \)
\end{itemize}

\subsection{Dipolo Hertziano}
\begin{itemize}
	\item Lunghezza efficace \(l_e = l\)
	\item Area efficace \(A_e = \dfrac{3}{8} \dfrac{\lambda^2}{\pi} \)
	\item Funzione di direttività \(f(\theta, \phi) = \sin^2(\theta) \)
	\item Tensione a vuoto \( V_0 = l_e \cdot E_{inc} \) 
	\item Densità di potenza \( S = \dfrac{P_t D}{4 \pi R^2} f(\theta, \phi) \)
	\item Resistenza di radiazione \( R_R = \dfrac{2}{3} \pi \eta_0 \left( \dfrac{l}{\lambda} \right) ^ 2 \)
	\item Campo elettrico irradiato (campo lontano) \( E_\theta = \dfrac{j \omega \mu I l}{4 \pi R} \exp{-j \beta R}\) con \(\theta\) colatitudine e \(R\) distanza tra punto considerato e centro del dipolo
\end{itemize}

\subsection{Spira magnetica}
\textbf{Ipotesi:} incidenza perpendicolare, adattamento di polarizzazione
\begin{itemize}
	\item Lunghezza efficace \(l_e = l_m \dfrac{Z_{in}}{\eta_0}\)
	\item Funzione di direttività \(f(\theta, \phi) = 1 \)
	\item Tensione a vuoto \( V_0 = j \omega \mu \dfrac{E_{inc}}{\eta_0} S \) 
	\item Densità di potenza \( S = \dfrac{P_t D}{4 \pi R^2} f(\theta, \phi) \)
	\item Resistenza di radiazione \( R_R = \eta_0 \dfrac{8 \pi^3}{3}\left( \dfrac{5}{\lambda^2} \right) ^ 2 \)
\end{itemize}

\subsection{Nastro di corrente}
\textbf{Ipotesi:} \(l_g = \) lunghezza fisica 
\begin{itemize}
	\item Lunghezza magnetica \( l_m = l_g \)
	\item Induttanza \( L = \mu_0 \dfrac{S}{l_g} \)
\end{itemize}

\subsection{Solenoide}
\textbf{Ipotesi:} \(l_g = \) lunghezza fisica, \(N\) spire su \(l_g\) 
\begin{itemize}
	\item Lunghezza magnetica \(l_m = \dfrac{l_g}{N} \)
	\item Induttanza \( L = \mu_0 N^2 \dfrac{S^2}{l_g} \)
\end{itemize}

\subsection{Confronto tra spira e solenoide}
\vspace{15pt}
\begin{center}
	\begin{tabular}{|Sc|Sc|Sc|}
		\hline 
		\textbf{in TX} & \textit{Spira} & \textit{Solenoide} \\ 
		\hline 	
		campo elettrico & \(E_0\) & \(N \cdot E_0\) \\ 
		\hline 
		campo magnetico & \(H_0\) & \(N \cdot H_0\) \\ 
		\hline 
		densità di potenza & \(S_0 = \dfrac{|E_0|^2}{2 \eta_0}\) & \(S_0 = \dfrac{|E|^2}{2 \eta_0} = N^2 S_0\) \\ 
		\hline 
		\(P_T\) & \(P_0\) & \(N^2 \cdot P_0\) \\ 
		\hline 
		\(R_R\) & \(R_{R0}\) & \(N^2 \cdot R_{R0}\)\\
		\hline
	\end{tabular} 
\end{center}
\vspace{5pt}
\begin{center}
	\begin{tabular}{|Sc|Sc|Sc|}
		\hline 
		\textbf{in RX} & \textit{Spira} & \textit{Solenoide} \\ 
		\hline 	
		tensione a vuoto & \(V_0\) & \(N \cdot V_0\) \\ 
		\hline 
		potenza disponibile & \(P_{D0}\) & \(P_{D0}\) \\ 
		\hline 
	\end{tabular} 
\end{center}
\vspace{5pt}
\end{document}